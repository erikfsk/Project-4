
\subsection{the Ising model}

The Ising model describes a coupled system. Where only the nearest neighbor affect each other. In this report the Ising model will be applied to a two dimensional magnetic system. This will be a grid of spins, where each spin $s_i$ can either have 1 or 0 as value. The total energy is expressed as: 


\begin{align*}
	E = - \sum_{<i,j>} J_{i,j} s_i s_j
\end{align*}

Where the symbol < kl > indicates that we sum over nearest neighbors only.
If we assume that each coupling has the same magnitude J, then the energy is expressed as: 

\begin{align}
	E = - J \sum_{<i,j>} s_i s_j
	\label{eq:ising-energy}
\end{align}

\subsubsection{Periodic boundary conditions}

When working with a finite matrix we run into a problem with the boundaries. They are missing neighbours. We solve this by introducing periodic boundary conditions. This means that the right neighbour for $S_n$ is assumed to take the value of $S_1$. 





















\subsection{Statistical physics}

\subsubsection{the partition function}\label{sec:z-func}

Boltzmann distribution is used as the probability distribution. Boltzmann distribution states the probability for $E_i$ is proportional to $e^{-\beta E_i}$ , where $\beta$ is $\frac{1}{k_BT}$. k is the Boltzmann constant. For this to be a probability distribution, it needs to be normalized. To normalize the distribution divide the sum of probabilities by a constant Z: 

\begin{align*}
	1 = \frac{\sum_{i} e^{-\beta E_i}}{Z}
	\\
	Z = \sum_{i} e^{-\beta E_i}
\end{align*}

Z is called the partition function. 

\subsubsection{Calculation of values}\label{sec:expect}

The partition function is very useful. In combination with the Boltzmann distribution we get a expression for the probability. 

\begin{align*}
	P(E_i) = \frac{e^{-\beta E_i}}{Z}
\end{align*}

For finding a mean value, one can simply make a sum over $P(E_i)$ multiplied by the value of interest. For instance the mean energy is given by: 

\begin{align*}
	\langle E \rangle = \sum_i E_i P(E_i)
\end{align*}

Expressions for important expectation values can be derived such for the energy E, magnetic moment |M|, specific heat capacity $C_v$ and the susceptibility $\chi$. The expressions used in this report are listed below\cite{compphys}:\footnote{lecture note page 420}

\begin{align}
	&\langle E \rangle = \sum_i E_iP(E_i)
	\label{eq:E}
	\\
	&\langle |M| \rangle = \sum_i M_iP(E_i)
	\label{eq:M}
	\\
	&\langle C_V \rangle = \frac{1}{k T^2}
	\left(
	\langle E^2 \rangle - \langle E \rangle ^2
	\right)
	\label{eq:cv}
	\\
	&\langle \chi \rangle = \frac{1}{kT} 
	\left(
	\langle M^2 \rangle - \langle |M| \rangle ^2
	\right)
	\label{eq:chi}
\end{align}












\subsection{Phase transition}

The two dimensional Ising model is able to predict a phase transition in the material. At a critical temperature $T_C$ the quantities for the material will start to behave differently. For $C_V$ and for $\chi$ the phase transition is a sharp peak when plotted against Temperature. For $|M|$ and E it can be seen, but only as a slight change in value. 
\\
\\
A second order phase transition is characterized by a correlation length. For finite lattice the correlation length is equal to the length of the system. $T_C$ can be obtain through scaling of the results from a finite system with a infinite system:

\begin{align}
    &T_C (L) - T_C (L=\infty) = a L^{\frac{-1}{v}}
    \label{eq:tc}
\end{align}

a is an unknown constant and v = 1. For finding a we use eq. \ref{eq:tc} with two different L. Substract the expression with $L_i$ by the expression with $L_j$ and we get: 

\begin{align*}
    &T_C (L_i) - T_C (L_j) = a 
    \left(
    L_i^{\frac{-1}{v}}-L_j^{\frac{-1}{v}}
    \right)
\end{align*}
\begin{align}
    &a = 
    \frac{T_C (L_i) - T_C (L_j)} 
    {
    L_i^{\frac{-1}{v}}-L_j^{\frac{-1}{v}}
    } \label{eq:find-a}
\end{align}

We combine this with eq. \ref{eq:tc} and we get an expression for $T_C(\infty)$:

\begin{align*}
    &T_C (L) - T_C (L=\infty) = a L^{\frac{-1}{v}}
\end{align*}
\begin{align}
    &T_C (L=\infty) = T_C (L) -  
    \frac
    {
    T_C (L_i) - T_C (L_j)
    } 
    {
    L_i^{\frac{-1}{v}}-L_j^{\frac{-1}{v}}
    } 
     L^{\frac{-1}{v}}
    \label{eq:t-c}
\end{align}












