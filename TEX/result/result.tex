\subsection{Analytic 2x2}


\subsubsection{Microstates 2x2}

\begin{center}
\label{tab:states-2x2}
\captionof{table}{This shows the different microstates that is possible for a 2x2 spinmatrix. It also states the energy and magnetic moment for each microstate.}
\begin{tabularx}{\textwidth}{c c c X c c c}
    \hline 
    \hline 
        State & Energy & Magnetic moment && State & Energy & Magnetic moment \\ 
    \hline
        \tilstand{1}{1}{1}{1} & -8J & 4 && \tilstand{0}{0}{0}{0} & -8J & -4 \\ \\
        
        \tilstand{0}{1}{1}{1} & 0J & 2 && \tilstand{1}{0}{0}{0} & 0J & -2 \\ \\
        \tilstand{1}{0}{1}{1} & 0J & 2 && \tilstand{0}{1}{0}{0} & 0J & -2 \\ \\
        \tilstand{1}{1}{0}{1} & 0J & 2 && \tilstand{0}{0}{1}{0} & 0J & -2 \\ \\
        \tilstand{1}{1}{1}{0} & 0J & 2 && \tilstand{0}{0}{0}{1} & 0J & -2 \\ \\

        \tilstand{0}{0}{1}{1} & 0J & 0 && \tilstand{1}{1}{0}{0} & 0J & 0 \\ \\ 
        \tilstand{0}{1}{0}{1} & 0J & 0 && \tilstand{1}{0}{1}{0} & 0J & 0 \\ \\
        \tilstand{1}{0}{0}{1} & 8J & 0 && \tilstand{0}{1}{1}{0} & 8J & 0 \\ \\
    \hline
\end{tabularx}
\end{center}



\begin{center}
\label{tab:states-2x2-summary}
\captionof{table}{The table shows a summary from table \ref{tab:states-2x2}. }
\begin{tabularx}{\textwidth}{c X c X c X c}
    \hline 
    \hline 
        Number of $\color{red}{\uparrow}$ && Multiplicity && Energy && Magnetic moment \\ 
    \hline
        4   &&      1      &&      -8J     &&       4       \\  
        3   &&      4      &&      0J      &&       2       \\
        2   &&      2      &&      8J      &&       0       \\
        2   &&      4      &&      0J      &&       0       \\
        1   &&      4      &&      0J      &&       -2      \\
        0   &&      1      &&      -8J     &&       -4      \\
    \hline
\end{tabularx}
\end{center}












\pagebreak
\subsubsection{Quantities}

We will use the equations from section \ref{sec:expect}.
\\
\\
For energy the eq. \ref{eq:E} will result in:

\begin{align*}
    &Z = \sum_i = e^{-\beta E_i}
    \\
    &\text{T = kT/J = 1} 
    \\
    &Z = \sum_i = e^{-\beta E_i} = 2e^{8} + 2e^{-8} + 12
\end{align*}


For energy the eq. \ref{eq:E} will give the result:

\begin{align*}
    &\langle E \rangle = \sum_i E_iP(E_i)
    \\
    &\text{T = kT/J = 1} 
    \\
    &\langle E \rangle = \frac{1}{Z} \sum_i E_i e^{-E_i}
    \\
    &\langle E \rangle = \frac{1}{Z} \left( -16 e^8 + 16e^{-8} \right) = -7.9839
    \\ 
    &\langle E \rangle /N= \frac{\langle E \rangle}{4} = -1.9959
\end{align*}


For energy the eq. \ref{eq:M} will give the result:

\begin{align*}
    &\langle |M| \rangle = \sum_i M_iP(E_i)
    \\
    &\text{T = kT/J = 1} 
    \\
    &\langle |M| \rangle = \frac{1}{Z} \sum_i M_i e^{-E_i}
    \\
    &\langle |M| \rangle 
    = 
    \frac{1}{Z} 
    \left(
      4\cdot1e^{8} 
    + 2\cdot4e^{0} 
    + 0\cdot2e^{-8} 
    + 0\cdot4e^{0} 
    + 2\cdot4e^{0}  
    + 4\cdot1e^{8}
    \right) 
    \\
    &\langle |M| \rangle 
    = 
    \frac{1}{Z} 
    \left(
    16
    + 8 e^{8}
    \right) = 3.9946
    \\ 
    &\langle |M| \rangle /N= \frac{\langle M \rangle}{4} = 0.9986
\end{align*}


\pagebreak
For $C_V$ we need to calculate $\langle E^2 \rangle$:

\begin{align*}
    &\langle E^2 \rangle = \sum_i E_iP(E_i)
    \\
    &\text{T = kT/J = 1} 
    \\
    &\langle E^2 \rangle = \frac{1}{Z} \sum_i E_i^2 e^{-E_i}
    \\
    &\langle E^2 \rangle = \frac{1}{Z} \left( 128 e^8 + 128 e^{-8} \right)
    \\
    &C_V = \langle E^2 \rangle - \langle E \rangle^2 = 0.12832
    \\ 
    &C_V/N = 0.03208
\end{align*}



For $\chi$ we need to calculate $\langle M^2 \rangle$:

\begin{align*}
    &\langle M^2 \rangle = \sum_i M_i^2P(E_i)
    \\
    &\text{T = kT/J = 1} 
    \\
    &\langle M^2 \rangle = \frac{1}{Z} \sum_i M_i^2 e^{-E_i}
    \\
    &\langle M^2 \rangle 
    = 
    \frac{1}{Z} 
    \left(
      16\cdot1e^{8} 
    + 4\cdot4e^{0} 
    + 0\cdot2e^{-8} 
    + 0\cdot4e^{0} 
    + 4\cdot4e^{0}  
    + 16\cdot1e^{8}
    \right) 
    \\
    &\langle M^2 \rangle 
    = 
    \frac{1}{Z} 
    \left(
    32
    + 32 e^{8}
    \right) = 15.9732
    \\
    &\langle \chi \rangle = 0.01604
    \\
    &\langle \chi \rangle / N = 0.004010
\end{align*}

Below you can see a summary for the quantities:

\begin{align*}
    \langle E \rangle /N &= -1.9959  \qquad &&\langle |M| \rangle /N = 0.9986
    \\
    C_V/N &= 0.03208  \qquad &&\langle \chi \rangle / N = 0.004010
\end{align*}
























\pagebreak
\subsection{Simulation 2x2}

These simulations ran for $10^5$ monte carlo cycles. All the simulations were done at T=1 and for a 2x2 grid.

\begin{figure}[H]
    \centering
    \begin{subfigure}{0.5\textwidth}
        \centering
        \includegraphics[width=\linewidth]{result/bilder/2x2/energy22}
        \caption{}
    \end{subfigure}%
    ~ 
    \begin{subfigure}{0.5\textwidth}
        \centering
        \includegraphics[width=\linewidth]{result/bilder/2x2/energyerror22}
        \caption{}
    \end{subfigure}
    \caption{a) }
    \label{fig:22-energy}
\end{figure}

\begin{figure}[H]
    \centering
    \begin{subfigure}{0.5\textwidth}
        \centering
        \includegraphics[width=\linewidth]{result/bilder/2x2/cv22}
        \caption{}
    \end{subfigure}%
    ~ 
    \begin{subfigure}{0.5\textwidth}
        \centering
        \includegraphics[width=\linewidth]{result/bilder/2x2/cverror22}
        \caption{}
    \end{subfigure}
    \caption{a) }
    \label{fig:22-energy}
\end{figure}

\begin{figure}[H]
    \centering
    \begin{subfigure}{0.5\textwidth}
        \centering
        \includegraphics[width=\linewidth]{result/bilder/2x2/mabs22}
        \caption{}
    \end{subfigure}%
    ~ 
    \begin{subfigure}{0.5\textwidth}
        \centering
        \includegraphics[width=\linewidth]{result/bilder/2x2/mabserror22}
        \caption{}
    \end{subfigure}
    \caption{a) }
    \label{fig:22-energy}
\end{figure}

\begin{figure}[H]
    \centering
    \begin{subfigure}{0.5\textwidth}
        \centering
        \includegraphics[width=\linewidth]{result/bilder/2x2/chi22}
        \caption{}
    \end{subfigure}%
    ~ 
    \begin{subfigure}{0.5\textwidth}
        \centering
        \includegraphics[width=\linewidth]{result/bilder/2x2/chierror22}
        \caption{}
    \end{subfigure}
    \caption{a) }
    \label{fig:22-energy}
\end{figure}





\subsection{Accepted configurations}

\begin{figure}[H]
    \centering
    \begin{subfigure}{0.5\textwidth}
        \centering
        \includegraphics[width=\linewidth]{result/bilder/config/energy22-MC1000000T1-configN20}
        \caption{}
    \end{subfigure}%
    ~ 
    \begin{subfigure}{0.5\textwidth}
        \centering
        \includegraphics[width=\linewidth]{result/bilder/config/energy22-MC1000000T1-config-RNGN20}
        \caption{}
    \end{subfigure}
    \caption{a) }
    \label{fig:config-T1}
\end{figure}

\begin{figure}[H]
    \centering
    \begin{subfigure}{0.5\textwidth}
        \centering
        \includegraphics[width=\linewidth]{result/bilder/config/energy22-MC1000000T24-configN20}
        \caption{}
    \end{subfigure}%
    ~ 
    \begin{subfigure}{0.5\textwidth}
        \centering
        \includegraphics[width=\linewidth]{result/bilder/config/energy22-MC1000000T24-config-RNGN20}
        \caption{}
    \end{subfigure}
    \caption{a) }
    \label{fig:config-T24}
\end{figure}



















\pagebreak
\subsection{Probability distribution}

Several simulations were made with different L. Below a selected few of these simulations are shown. All the data and figures for this section are available at my \href{https://github.com/erikfsk/Project-4/tree/master/Project4/Result/4d/}{\textcolor{blue}{github}}. The figures below were created with a python \href{https://github.com/erikfsk/Project-4/blob/master/Project4/Result/4d/plot-hist.py}{\color{blue}{script}}.

\begin{figure}[H]
    \centering
    \begin{subfigure}{0.5\textwidth}
        \centering
        \includegraphics[width=\linewidth]{result/bilder/hist/MC1000000T1-distN20-hist}
        \caption{}
    \end{subfigure}%
    ~ 
    \begin{subfigure}{0.5\textwidth}
        \centering
        \includegraphics[width=\linewidth]{result/bilder/hist/MC1000000T24-distN20-hist}
        \caption{}
    \end{subfigure}
    \caption{a) }
    \label{fig:tc-chi-cv}
\end{figure}























\pagebreak
\subsection{Phase transition}



\subsubsection{Numerical studies of phase transition}

Several simulations were made with different L. Below a selected few of these simulations are shown. All the data and figures for this section are available at my \href{https://github.com/erikfsk/Project-4/tree/master/Project4/Result/4e/}{\textcolor{blue}{github}}. The figures below were created with a python \href{https://github.com/erikfsk/Project-4/blob/master/Project4/Result/4e/plot-4e.py}{\color{blue}{script}}.

\begin{figure}[H]
    \centering
    \begin{subfigure}{0.5\textwidth}
        \centering
        \includegraphics[width=\linewidth]{result/bilder/Tc/chi-Tc}
        \caption{}
    \end{subfigure}%
    ~ 
    \begin{subfigure}{0.5\textwidth}
        \centering
        \includegraphics[width=\linewidth]{result/bilder/Tc/cv-Tc}
        \caption{}
    \end{subfigure}
    \caption{a) }
    \label{fig:tc-chi-cv}
\end{figure}

\begin{figure}[H]
    \centering
    \begin{subfigure}{0.5\textwidth}
        \centering
        \includegraphics[width=\linewidth]{result/bilder/Tc/e-Tc}
        \caption{}
    \end{subfigure}%
    ~ 
    \begin{subfigure}{0.5\textwidth}
        \centering
        \includegraphics[width=\linewidth]{result/bilder/Tc/m-Tc}
        \caption{}
    \end{subfigure}
    \caption{a) }
    \label{fig:tc-E-M}
\end{figure}

\subsubsection{Extracting the critical temperature}


\begin{align*}
    &T_C (L) - T_C (L=\infty) = a L^{\frac{-1}{v}}
\end{align*}

\begin{align*}
    &T_C (L_i) - T_C (L_j) = a 
    \left(
    L_i^{\frac{-1}{v}}-L_j^{\frac{-1}{v}}
    \right)
    \\
    &a = 
    \frac{T_C (L_i) - T_C (L_j)} 
    {
    L_i^{\frac{-1}{v}}-L_j^{\frac{-1}{v}}
    }
\end{align*}


\begin{center}
\label{tab:extracting a}
\captionof{table}{The table shows how a differ for which $L_i$ and $L_j$ one uses. The values for $T_C$ were picked from figure \ref{fig:tc-chi-cv} a)}
\begin{tabularx}{\textwidth}{c X c X c X c X c}
    \hline 
    \hline 
        $L_i$ && $L_j$ && $T_{C_i}$ && $T_{C_j}$ && a\\ 
    \hline
        60      &&      100     &&  2.30  &&  2.29  && 0.00025 \\
        60      &&      140     &&  2.30  &&  2.28  && 0.00025 \\
        60      &&      200     &&  2.30  &&  2.27  && 0.0002142 \\
        100     &&      140     &&  2.29  &&  2.28  && 0.00025 \\
        100     &&      200     &&  2.29  &&  2.27  && 0.0002 \\
        140     &&      200     &&  2.28  &&  2.27  && 0.0001666 \\
    \hline
\end{tabularx}
\end{center}

From the values for a above, we can extract the average value of a.
 $\overline{a}$ turns out to be 0.0002218.

\begin{align*}
    &T_C (L) - T_C (L=\infty) = a L^{\frac{-1}{v}}
    \\
    &T_C (L=\infty) =T_C (L) -  a L^{\frac{-1}{v}}
\end{align*}

\begin{center}
\label{tab:extracting a}
\captionof{table}{The table shows how a differ for which $L_i$ and $L_j$ one uses. The values for $T_C$ were picked from figure \ref{fig:tc-chi-cv} a)}
\begin{tabularx}{\textwidth}{c X c X c X c }
    \hline 
    \hline 
        $L_i$ && $T_C(L_i)$ && $T_C(\infty)$ with $\overline{a}$ \\ 
    \hline
        60      &&  2.30  && 2.2999\\
        100     &&  2.29  && 2.2899\\
        140     &&  2.28  && 2.2799\\
        200     &&  2.27  && 2.2699\\
    \hline
\end{tabularx}
\end{center}
